\documentclass{ltjsarticle}
\usepackage{hyperref} %リンクを有効にする

\usepackage[utf8]{inputenc}
\hypersetup{
    colorlinks=true,
    citecolor=blue,
    linkcolor=blue,
}

\usepackage{xcolor}
\usepackage{ulem}

\usepackage{tabularx}
\usepackage{booktabs}

\usepackage{url} %同上
\usepackage{csvsimple}
\usepackage[dvipdfmx]{graphicx}

\usepackage{autobreak}
\usepackage{mathtools}
\usepackage[T1]{fontenc}

\usepackage{lmodern}
\usepackage{amsmath,amssymb} %もちろん

\usepackage{amsfonts,amsthm,mathtools} %もちろん
\usepackage{braket,physics} %あると便利なやつ
\usepackage{bm} %ラプラシアンで使った
\usepackage[top=30truemm,bottom=30truemm,left=20truemm,right=20truemm]{geometry} %余白設定
\usepackage{latexsym}

%ごくたまに必要になる
\newcommand{\R}{\mathbb{R}}
\newcommand{\Z}{\mathbb{Z}}
\newcommand{\N}{\mathbb{N}}
\newcommand{\C}{\mathbb{C}} 
\usepackage{comment}
\usepackage{pdfpages}

\usepackage{here}
\usepackage{comment}
\usepackage{multicol}
\usepackage{multicol}
\setlength{\columnsep}{5mm}
\columnseprule=0.2mm
\mathtoolsset{showonlyrefs=true}

%当然のようにやる.
\allowdisplaybreaks[4]
%ここまで今回の記事関係ない
\usepackage{tcolorbox}
\tcbuselibrary{breakable, skins, theorems}

\newtcbtheorem{theorem1}{定理}{enhanced,  %TikZの内部処理を導入する.ある程度複雑なものには必須.
    attach boxed title to top left = {xshift=5mm,yshift=-3mm},
    boxed title style = {colframe = black!35!black, colback = white},
    coltitle = black,
    colback = white,
    colframe = black!35!black,
    fonttitle = \bfseries,
    breakable = true,
    top = 4mm
}{korehatheorem1}


\newtcbtheorem{theorem2}{定義}{enhanced,  %TikZの内部処理を導入する.ある程度複雑なものには必須.
    attach boxed title to top left = {xshift=5mm,yshift=-3mm},
    boxed title style = {colframe = black!35!black, colback = white},
    coltitle = black,
    colback = white,
    colframe = black!35!black,
    fonttitle = \bfseries,
    breakable = true,
    top = 4mm
}{korehatheorem2}

\newtcbtheorem{theorem3}{定義}{enhanced,  %TikZの内部処理を導入する.ある程度複雑なものには必須.
    attach boxed title to top left = {xshift=5mm,yshift=-3mm},
    boxed title style = {colframe = black!35!black, colback = white},
    coltitle = black,
    colback = white,
    colframe = black!35!black,
    fonttitle = \bfseries,
    breakable = true,
    top = 4mm
}{korehatheorem3}

\usepackage{ascmac}
\usepackage{array}

\newcommand{\prefacename}{Preface}
\newenvironment{preface}{
    \vspace*{\stretch{2}}
    {\noindent \bfseries \Huge \prefacename}
    \begin{center}
        % \phantomsection \addcontentsline{toc}{chapter}{\prefacename} % enable this if you want to put the preface in the table of contents
        \thispagestyle{plain}
    \end{center}%
}
{\vspace*{\stretch{5}}}

\renewcommand{\baselinestretch}{1.5}
\begin{comment}
    \usepackage{makeidx}
    \makeindex
\end{comment}

\usepackage{enumitem}


\title{卒論 ノート}
\author{氏名:久野証\\所属:東大工学部計数工学科数理情報工学コース\\学籍番号: 03-210599}
\date{\today}

\begin{document}
  \maketitle
  \tableofcontents
  \begin{abstract}
    2023年Aセメスターの卒論執筆に際して、勉強したことや考えたことをここにまとめた。
  \end{abstract}
  \clearpage

  
  \section{Papers}
\subsection{Emergence of a resonance in machine learning}
Zheng-Meng Zhai , 1 Ling-Wei Kong , 1 and Ying-Cheng Lai 1,2,*
1School of Electrical, Computer and Energy Engineering, Arizona State University, Tempe, Arizona 85287, USA
2Department of Physics, Arizona State University, Tempe, Arizona 85287, USA.

\noindent (Received 9 June 2022; revised 1 March 2023; accepted 26 July 2023; published 24 August 2023)
\\

\noindent キーワード:Resonance in nonlinear dynamical systems, 

\subsubsection{要旨}

\begin{enumerate}
  \item 入力信号にノイズを挿れた場合のReservoir Computingを考える。\begin{enumerate}
    \item hyperparametersが最適化されていない時でも、ノイズを挿れることで予測の精度をあげることができる。
    \item もっとも良い精度を達成するには、hyperparametersが最適化されていなければならない。\begin{enumerate}
      \item Hyperparametersに対するBaysian optimizationで最適化可能。
      \item 確率共鳴があると決定づけるために、ノイズの振幅をhypermparameterに数える。
    \end{enumerate}
  \end{enumerate} 
  \item Macky-Glass (MG) systemとKuramoto-Sivashinsky (KS) systemに対して、シミュレーションを行う。
  \item 物理側から、確率共鳴が生まれる原理を考える。
\end{enumerate}
\clearpage

\subsubsection{状況設定}
Appendix A 参照。
\begin{enumerate}
  \item hyperparameters:\begin{enumerate}
    \item $\rho$: the specral radius of the  reservoir network.
    \item $\gamma$: the scaling factor of the input weights.
    \item $\alpha$: the leakage parameter 
    \item $\beta$: the regularization coefficient 
    \item p: the link connection probability of the random network in the hidden layer.
    \item $\sigma$: the noise amplitude 
  \end{enumerate}
  \item hyperparametersの最適化\begin{enumerate}
    \item MATLAB: SURROGATEOPT を用いる。
      \footnote{"The Bayesian
      optimization method can be implemented using PYTHON or
      other languages. Different packages for Bayesian optimization are now available, such as BAYESIAN-OPTIMIZATION and
      BOTORCH in PYTHON."としている。}
    \item $\sigma$ごとにhyperparametersの最適化を行うので、$\sigma$に対する他のhyperparametersの組みは異なる。
  \end{enumerate}
  \item シミュレーションを行う。
  \begin{enumerate}
    \item MG system: 
    $$\dot{s}(t)=a s(t-\tau) /\left(1+[s(t-\tau)]^c\right)-b s(t),$$ $\tau$ is the time delay, $a, b$, and $c$ are parameters.
    \footnote{The state of the system at time t is determined by the entire prior state history
    within the time delay, making the phase space of the system
    infinitely dimensional.}  
    \begin{enumerate}
      \item $a=0.2, b=0.1, \text { and } c=10$を固定。
      \item $\tau=17,\ \text{Lyapunov exponents: } \lambda_{+} \approx 0.006 \text { と } \tau=30,\ \text{Lyapunov exponents: } \lambda_{+} \approx 0.011 \text{ and } 0.003$の2つの場合を比べる.
      \item $\Delta t=100 h=1.0.$
      \item Warmup: $10000 \Delta t.$
      \item 時系列データには事前にz-score normalization: $z(t)=[s(t)-\bar{s}] / \sigma_s$を施す。
    \end{enumerate}
  \end{enumerate}
\end{enumerate}
\clearpage

\subsubsection{面白いと思ったところ}
\begin{enumerate}
  \item Introductionでも述べられているが、ノイズを入れることによって、初期値鋭敏性を持つカオスシステムに対して。短期的にも長期的にも予測の精度を挙げられるということ。
\end{enumerate}

\subsubsection{論文を受けての今後の研究方向}
\subsubsection{疑問点}
\begin{enumerate}
  \item Fig.4とFig.5について,MG system における$\tau$の値が30から17に変えると,$\sigma$にどのような影響があるか.なぜその影響が生まれるか.
  \item なぜ,Fig.2の(逆)ピークを与える$\sigma$帯とFig.6(c)の(逆)ピークを与える$\sigma$帯が重なるのか.
  \item IIIで,Machine learningにおけるresonanceが生まれるPhysical reasonを挙げているが,これは対象を正しく説明できているか.extraordinarily complicatedなhidden layerの中身を解析することなく,physical reasonを与えることが,なにを説明しているのか/なにを説明していないのか.
\end{enumerate}

\subsubsection{関連する文献}
 
\clearpage
\subsubsection{用語まとめ}

\subsubsection{Abstract}
\begin{enumerate}
  \item stochastic/coherence resonance: 
  \item nonlinear dynamical system:
  \item regularizer/regularization:
  \item reservoir computing:
  \item state variables/attractor:
  \item hyperparameters:
\end{enumerate}
\subsubsection{I. Introduction}
\begin{enumerate}
  \item model-free/data-driven:
  \item oscillatoin/Lyapnov times:
  \item trajectory:
  \item basin boundary: 
  \item robustness:
  \item Baysian optimization:
\end{enumerate}

\subsubsection{II. Result}
\begin{enumerate}
  \item SURROGATEOPT function (MATLAB):
  \item surrogate approximation function:
  \item objective function:
  \item global minimum:
  \item sampling/updating:
  \item radial basis function: 
  \item Mackey-Glass (MG) system:
  \item spatiotemporal chaotic Kuramoto-Sivashinsky (KS) system: 
\end{enumerate}

\subsubsection{A. Emergence of a resonance from short-term prediction}
\begin{enumerate}
  \item transient behavior:
  \item z-score normalization: 
  \item periodic boundary condition:
  \item Prediction horizon/stability: 
\end{enumerate}

\subsubsection{B. Emergence of a resonance from long-term prediction}
\begin{enumerate}
  \item collapse:
  \item wider/narrower resonance:
\end{enumerate}

\subsubsection{III. HEURISTIC REASON FOR THE OCCURRENCE
OF A RESONANCE}
\begin{enumerate}
  \item time-scale match:
  \item the mean first-passage time:
  \item nonlinear activation: 
  \item linear reservoir computing:
  \item noise-enhanced temporal regularity:
  \item vector autoregressive process (VAR): 
\end{enumerate}

\subsubsection{IV. DISCUSSION}
\begin{enumerate}
  \item magnitude: 
\end{enumerate}

\subsubsection{Appendix A}
\begin{enumerate}
  \item recurrent neural network(RNN):
  \item input/hidden/output layer: 
  \item linear regression: 
  \item adjacency matrix:
  \item state vector: 
  \item dynamical state/evolution: 
  \item neuron:
  \item leakage parameter $\alpha$:
  \item link probability p: 
  \item spectral radius:
\end{enumerate}

\clearpage
  \section{Lectures}

\begin{enumerate}
    \item Reservoir Computing\begin{enumerate}
        \item \href{https://www.youtube.com/watch?v=AhxROWo_FOw}{Reservoir Computing for SDEs(Josef Teichmann:)}
        \item \href{https://www.youtube.com/watch?v=lak3OjvE_44}{Reservoir Computing \& Dynamical Systems - Second Sumposium on Machine Learning and Dynamical Systems(Josef Teichmann)}
        \item \href{https://www.youtube.com/watch?v=wbH4En-k5Gs}{Introduction to Next Generation Reservoir Computing(Daniel Gauthier)}
    \end{enumerate}
    \item Machine Learning in general\begin{enumerate}
        \item \href{https://people.math.ethz.ch/~jteichma/index.php?content=teach_mlf2023}{Machine Learning in Finance(Josef Teichmann)}
    \end{enumerate}
\end{enumerate}

\subsection{Josef Teichmann: Reservoir Computing for SDEs}
\href{https://www.youtube.com/watch?v=AhxROWo_FOw}{Access from here.}
\begin{enumerate}
    \item We consider differential equations of the form
    $$
    d Y_t=\sum_i V_i\left(Y_t\right) d u_t^i, Y_0=y \in E
    $$
    to construction evolutions in state space $E$ (could be a manifold of finite or infinite dimension) depending on local characteristics, initial value $y \in E$ and the control $u$.
    \item Theorem (Universality)
    Let Evol be a smooth evolution operator on a convenient vector space $E$ which satisfies (again the time derivative is taken with respect to the forward variable $t$ ) a controlled ordinary differential equation
    $$
    d \mathrm{Evol}_{s, t}(x)=\sum_{i=1}^d V_i\left(\mathrm{Evol}_{s, t}(x)\right) d u^i(t) .
    $$
    Then for any smooth (test) function $f: E \rightarrow \mathbb{R}$ and for every $M \geq 0$ there is a time-homogenous linear $W=W\left(V_1, \ldots, V_d, f, M, x\right)$ from $\mathbb{A}_d^M$ to the real numbers $\mathbb{R}$ such that
    $$
    f\left(\operatorname{Evol}_{s, t}(x)\right)=W\left(\pi_M\left(\operatorname{Sig}_{s, t}(1)\right)\right)+\mathcal{O}\left((t-s)^{M+1}\right)
    $$
    for $s \leq t$
    \item Signature as universal dynamical system \begin{enumerate}
        \item This explains that any solution can be represented - up to a linear readout - by a universal reservoir, namely signature. Similar constructions can be done in regularity structures, too (branched rough paths, etc).
        \item This is used in many instances of provable machine learning by, e.g., groups in Oxford (Harald Oberhauser, Terry Lyons, etc), and also ...
        \item ... at JP Morgan, in particular great recent work on 'Nonparametric pricing and hedging of exotic derivatives' by Terry Lyons, Sina Nejad and Imanol Perez Arribas.
        \item in contrast to reservoir computing: signature is high dimensional (i.e. infinite dimensional) and a precisely defined, non-random object.
        \item Can we approximate signature by a lower dimensional random object with similar properties?
    \end{enumerate}
\end{enumerate}

\clearpage
  \section{TODOs}
\begin{enumerate}
  \item \href{https://github.com/reservoirpy/reservoirpy}{reservoirpy関連}
  \begin{enumerate}
  \item \sout{Week 2: \href{https://github.com/reservoirpy/reservoirpy/blob/master/tutorials/4-Understand_and_optimize_hyperparameters.ipynb}{Understand and optimize ESN hyperparameters}などのTutorialページを読む.}
  \end{enumerate}
  \item Reservoir Computingについて学ぶ.\begin{enumerate}
    \item \sout{Week 2: レクチャーノートなどを通じて、知識を準備する。}とりあえずざっとは見た。
  \end{enumerate}
  \item \sout{Week 2 Github環境を整備する。} 9/28, Week 2: branchの作り方がよくわからないが、とりあえずpushは出来た。
  \item 論文\ref{Emergence of a resonance in machine learning}関連
    \begin{enumerate}
    \item \sout{9/28, Week 2: Bollt$[40]$を読む。}一応終わったが、あまり理解できていない。
    \item Week 3: Bollt$[40]$をもう一度読む。
    \item Week 3: Reservoir Computerに関する具体的な問題を見つけるため、情報収集する。
    \item Week 3: とりあえず何か実装してみる。\begin{enumerate}
      \item 9/28, Week 2: DVの実装。
      \item rx1を用いて、並列で動かしてみる。\begin{enumerate}
        \item 並列といってもどこを並列処理させる?異なる$\sigma$に対する他のhyperparametersの最適値を求めるとき?
        \item $\sigma$を動かしてみる。
      \end{enumerate}
    \end{enumerate}
    \item Week 3: ノイズの入れ方を工夫してみる。
    \item Week 3: ローレンツモデルに対してRCを用いていみる。
  \end{enumerate}
  \item Week 3: 今後の研究対象を決める。\begin{enumerate}
    \item 例えば、論文\ref{Emergence of a resonance in machine learning}から着想を得て、カオスに対するRCを考察するとする。
    しかし、カオスに対する機械学習の問題意識が予測精度の向上(と計算時間の短縮である)とするならば、その意味で新たに研究すべき対象は見つけにくいのではないか。
    \item 論文\ref{Bollt_paper}にあるとおり、RCはより広いシステムに対する研究も行われているようである。ひとまず、そこまで考える対象をある程度広げてしまった方が出発しやすいのではないだろうか。
    \item 一貫して機械学習の素養が足りていないため、機械学習そのものの内容というよりは、今は物理・数学的な話題から刺激/触発されたRCという文脈で考えておきたい(そのうちRCや機械学習自体の知識が蓄えられ、少し状況が変わるかもしれない)。
  \end{enumerate}
\end{enumerate}

\end{document}