\section{Papers}
\subsection{Emergence of a resonance in machine learning}
Zheng-Meng Zhai , 1 Ling-Wei Kong , 1 and Ying-Cheng Lai 1,2,*
1School of Electrical, Computer and Energy Engineering, Arizona State University, Tempe, Arizona 85287, USA
2Department of Physics, Arizona State University, Tempe, Arizona 85287, USA.

\noindent (Received 9 June 2022; revised 1 March 2023; accepted 26 July 2023; published 24 August 2023)
\\

\noindent キーワード:Resonance in nonlinear dynamical systems, 

\subsubsection{要旨}

\begin{enumerate}
  \item 入力信号にノイズを挿れた場合のReservoir Computingを考える。\begin{enumerate}
    \item hyperparametersが最適化されていない時でも、ノイズを挿れることで予測の精度をあげることができる。
    \item もっとも良い精度を達成するには、hyperparametersが最適化されていなければならない。\begin{enumerate}
      \item Hyperparametersに対するBaysian optimizationで最適化可能。
      \item 確率共鳴があると決定づけるために、ノイズの振幅をhypermparameterに数える。
    \end{enumerate}
  \end{enumerate} 
  \item Macky-Glass (MG) systemとKuramoto-Sivashinsky (KS) systemに対して、シミュレーションを行う。
  \item 物理側から、確率共鳴が生まれる原理を考える。
\end{enumerate}
\clearpage

\subsubsection{状況設定}
Appendix A 参照。
\begin{enumerate}
  \item hyperparameters:\begin{enumerate}
    \item $\rho$: the specral radius of the  reservoir network.
    \item $\gamma$: the scaling factor of the input weights.
    \item $\alpha$: the leakage parameter 
    \item $\beta$: the regularization coefficient 
    \item p: the link connection probability of the random network in the hidden layer.
    \item $\sigma$: the noise amplitude 
  \end{enumerate}
  \item hyperparametersの最適化\begin{enumerate}
    \item MATLAB: SURROGATEOPT を用いる。
      \footnote{"The Bayesian
      optimization method can be implemented using PYTHON or
      other languages. Different packages for Bayesian optimization are now available, such as BAYESIAN-OPTIMIZATION and
      BOTORCH in PYTHON."としている。}
    \item $\sigma$ごとにhyperparametersの最適化を行うので、$\sigma$に対する他のhyperparametersの組みは異なる。
  \end{enumerate}
  \item シミュレーションを行う。
  \begin{enumerate}
    \item MG system: 
    $$\dot{s}(t)=a s(t-\tau) /\left(1+[s(t-\tau)]^c\right)-b s(t),$$ $\tau$ is the time delay, $a, b$, and $c$ are parameters.
    \footnote{The state of the system at time t is determined by the entire prior state history
    within the time delay, making the phase space of the system
    infinitely dimensional.}  
    \begin{enumerate}
      \item $a=0.2, b=0.1, \text { and } c=10$を固定。
      \item $\tau=17,\ \text{Lyapunov exponents: } \lambda_{+} \approx 0.006 \text { と } \tau=30,\ \text{Lyapunov exponents: } \lambda_{+} \approx 0.011 \text{ and } 0.003$の2つの場合を比べる.
      \item $\Delta t=100 h=1.0.$
      \item Warmup: $10000 \Delta t.$
      \item 時系列データには事前にz-score normalization: $z(t)=[s(t)-\bar{s}] / \sigma_s$を施す。
    \end{enumerate}
  \end{enumerate}
\end{enumerate}
\clearpage

\subsubsection{面白いと思ったところ}
\begin{enumerate}
  \item Introductionでも述べられているが、ノイズを入れることによって、初期値鋭敏性を持つカオスシステムに対して。短期的にも長期的にも予測の精度を挙げられるということ。
\end{enumerate}

\subsubsection{論文を受けての今後の研究方向}
\subsubsection{疑問点}
\begin{enumerate}
  \item Fig.4とFig.5について,MG system における$\tau$の値が30から17に変えると,$\sigma$にどのような影響があるか.なぜその影響が生まれるか.
  \item なぜ,Fig.2の(逆)ピークを与える$\sigma$帯とFig.6(c)の(逆)ピークを与える$\sigma$帯が重なるのか.
  \item IIIで,Machine learningにおけるresonanceが生まれるPhysical reasonを挙げているが,これは対象を正しく説明できているか.extraordinarily complicatedなhidden layerの中身を解析することなく,physical reasonを与えることが,なにを説明しているのか/なにを説明していないのか.
\end{enumerate}

\subsubsection{関連する文献}
 
\clearpage
\subsubsection{用語まとめ}

\subsubsection{Abstract}
\begin{enumerate}
  \item stochastic/coherence resonance: 
  \item nonlinear dynamical system:
  \item regularizer/regularization:
  \item reservoir computing:
  \item state variables/attractor:
  \item hyperparameters:
\end{enumerate}
\subsubsection{I. Introduction}
\begin{enumerate}
  \item model-free/data-driven:
  \item oscillatoin/Lyapnov times:
  \item trajectory:
  \item basin boundary: 
  \item robustness:
  \item Baysian optimization:
\end{enumerate}

\subsubsection{II. Result}
\begin{enumerate}
  \item SURROGATEOPT function (MATLAB):
  \item surrogate approximation function:
  \item objective function:
  \item global minimum:
  \item sampling/updating:
  \item radial basis function: 
  \item Mackey-Glass (MG) system:
  \item spatiotemporal chaotic Kuramoto-Sivashinsky (KS) system: 
\end{enumerate}

\subsubsection{A. Emergence of a resonance from short-term prediction}
\begin{enumerate}
  \item transient behavior:
  \item z-score normalization: 
  \item periodic boundary condition:
  \item Prediction horizon/stability: 
\end{enumerate}

\subsubsection{B. Emergence of a resonance from long-term prediction}
\begin{enumerate}
  \item collapse:
  \item wider/narrower resonance:
\end{enumerate}

\subsubsection{III. HEURISTIC REASON FOR THE OCCURRENCE
OF A RESONANCE}
\begin{enumerate}
  \item time-scale match:
  \item the mean first-passage time:
  \item nonlinear activation: 
  \item linear reservoir computing:
  \item noise-enhanced temporal regularity:
  \item vector autoregressive process (VAR): 
\end{enumerate}

\subsubsection{IV. DISCUSSION}
\begin{enumerate}
  \item magnitude: 
\end{enumerate}

\subsubsection{Appendix A}
\begin{enumerate}
  \item recurrent neural network(RNN):
  \item input/hidden/output layer: 
  \item linear regression: 
  \item adjacency matrix:
  \item state vector: 
  \item dynamical state/evolution: 
  \item neuron:
  \item leakage parameter $\alpha$:
  \item link probability p: 
  \item spectral radius:
\end{enumerate}

\clearpage