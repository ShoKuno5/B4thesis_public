\documentclass[../main]{subfiles}
\begin{document}
\chapter{おわりに}
\label{chap:summary}
\section{本研究のまとめ}
本論文では,第\ref{chap:method-3body}章で,振動子ネットワークにおける枝の生成・消滅に伴う同期状態の変化を調べた.
特に,同期クラスタが変化する鞍替え現象に着目し,簡略化した数理モデルである3体系を提案した.
鞍替えする振動子集団と鞍替え前後の集団の3種類の振動子に注目し各集団について平均化を行うと,3つの振動子が鞍替えする振動子を中央にして鎖状に結合したモデルで近似される.
この3体系はネットワークの構造を固定し結合強度を変化させた場合,最大で5種類の同期状態を経ることがわかった.
そして,同期状態が変化する臨界結合強度の近似値を求めた.
解析の結果の以下のことが分かった.固有振動数が近く枝が多く集団のサイズが小さいほど同期しやすい.
そして,2つの集団との間の枝の本数の比がある程度偏っていて,結合強度がある程度強いとき鞍替え現象が起こる.
また,結合強度比を固定し結合強度を変化させたときの同期状態の変化パターンに関して,結合強度比$a$には少なくとも2つの分岐点が存在することが分かった.特に,それぞれの近似値$a^\ast_1,\ a^\ast_2$を求めた.
$0<a\lesssim a^\ast_1$のとき全部で3つの同期状態が観測される.そして,$a^\ast_1\lesssim a\lesssim a^\ast_2$では全部で5つの同期状態が認められ,$a\gtrsim a^\ast_2$では全部で3つの同期状態が見られる.
このように,3体系へ近似する手法は,鞍替え現象など2種類の振動子では説明できないような性質を示すようなネットワークの一部を説明できる.

続く第\ref{chap:percolation}章では,振動子ネットワークにおいて枝数を変化させる操作を複数回行った平均的なクラスタ状態について調べた,
特に,同じ振動数をもつ振動子を同一のクラスタに属するとしたときの最大のクラスタサイズとオーダーパラメータの変化について注目した.
最大クラスタは枝が少ないとき,孤立した振動子から構成され,枝が増えると最大の連結成分に属する振動子から構成される.
そして,異なる固有振動数を持つ振動子が同期するほど枝の本数が大きくなると,ネットワーク全体が同期し,ネットワーク全体が1つのクラスタとなる.
クラスタの最大サイズが2つの臨界本数に応じて変動するのに伴って,オーダーパラメータも変化し変曲点を2つもつ.

これらの同期状態の変化において,境界となる枝の臨界本数についてそれぞれ求めた.
このうち全体が同期する臨界本数は,結合強度が十分弱い場合について求めた.結合強度が強く,固有振動数の差が小さいほど小さくなることがわかった.
また.最大クラスタが孤立振動子から構成される臨界本数は,結合強度が十分弱い場合では孤立振動子がほとんどなくなるネットワークのサイズになることがわかった.
さらにネットワークのサイズが十分大きい場合について調べ,結合強度がある程度大きくなると,臨界本数が減少し,percolation が引き起こされるネットワークのサイズの半分の本数まで減少することがわかった.
このように,ネットワークにおけるクラスタサイズは,同期状態を調べるために用いられてきたオーダーパラメータよりも,顕著にネットワークにおけるミクロなダイナミクスを反映している.
したがって,ネットワークの構造の変化に伴う同期状態の変化を調べる上でクラスタサイズに注目することは有用であると考えられる.
\section{今後の課題}
本研究では,ネットワークとして Erd\H{o}s–R\'{e}nyi モデル,振動数分布として二項分布を用いた.
しかし,現実のネットワークのようなスケールフリー性などの特徴を Erd\H{o}s–R\'{e}nyi モデルは持たない.
よって,ネットワークの構造が機能に影響する振動子ネットワークについて考える場合は,より現実のネットワークに即したモデル\cite{Moreno_2004}を用いることが期待される.
また,振動数分布も二項分布以外に,よく用いられる連続分布\cite{RevModPhys.77.137}や時間変動する分布\cite{CUMIN2007181}を考慮することで.様々な同期現象についてネットワーク構造との関係を調べることができる.

\ref{sec:3body-critical}節では,解析を容易にするため3体系における2つの振動数差が大きく異なるという仮定を行った.
しかし,結合強度比を固定し結合強度を変化させたときの同期状態の変化パターンについて一般的な性質を知るには不十分である.
実際,\ref{sec:3body-discussion}節で述べたように,3体がどれも非同期の状態から3体が同期する必要十分条件は,本研究の手法では調べることができなかった.
この条件が正確に求まると,一般のネットワークの同期状態の変化パターンを解析する場合に3体系という最も簡略化された系に帰着させることの正当性がより明らかになる.
このため,今後はより一般のパラメータ領域において同期状態の変化パターンなどの3体系の性質を調べることが求められる.

\ref{sec:percolate-m}節では,解析のため,結合強度が十分弱いという仮定や,ネットワークのサイズが十分大きいという仮定を課した.
しかし,それ以外のパラメータ領域でも解析を行うことが,ネットワークの構造と同期状態の関係を理解する上で期待される.
\end{document}