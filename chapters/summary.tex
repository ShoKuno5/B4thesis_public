\documentclass[../main]{subfiles}
\begin{document}
\chapter{おわりに}
\label{chap:summary}
\section{本研究のまとめ}
本論文では,第\ref{chap:method-3body}章で,振動子ネットワークにおける枝の生成・消滅に伴う同期状態の変化を調べた.
特に,同期クラスタが変化する鞍替え現象に着目し,簡略化した数理モデルを提案し,同期状態が変化する臨界結合強度を求めた.
鞍替え現象は鞍替えする振動子と鞍替え前後の集団の3種類の振動子に注目し各集団について平均化を行うと,3つの振動子が鞍替えする振動子を中央にして鎖状に結合したモデルで近似される.
解析の結果の以下のことが分かった,固有振動数が近く枝が多く集団のサイズが小さいほど同期しやすい.
また,2つ集団との間の枝の本数の比がある程度偏っていて,結合強度がある程度強いとき鞍替え現象が起こる.

続く第\ref{chap:percolation}章では,振動子ネットワークにおいて枝数を変化させる操作を複数回行った平均的なクラスタ状態について調べた,
特に,同じ振動数をもつ振動子を同一のクラスタに属するとしたときの最大のクラスタサイズとオーダーパラメータの変化について注目した.
最大クラスタは枝が少ないとき,孤立した振動子から構成され,枝が増えると最大の連結成分に属する振動子から構成される.
そして,異なる固有振動数を持つ振動子が同期するほど枝の本数が大きくなると,ネットワーク全体が同期し,ネットワーク全体が1つのクラスタとなる.
これらの境界となる枝の臨界本数についてそれぞれ求めた.
このうち全体同期する臨界本数は結合強度が大きくなるほど少なくなるが,それより小さい臨界本数は結合強度に依らないネットワークの構造に依存したものであることが分かった.
\section{今後の課題}
本研究では,ネットワークとして Erdős–Rényi モデル,振動数分布として二項分布を用いた.
しかし,現実のネットワークのようなスケールフリー性などの特徴を Erdős–Rényi モデルは持たない.
よって,ネットワークの構造が機能に影響する振動数ネットワークについて考えるためには,より現実のネットワークに即したモデルを用いる必要がある.

また,振動数分布も……
\end{document}