\documentclass[../main]{subfiles}
\begin{document}
\chapter{はじめに}
\section{背景}
\label{chap:intro-back}
同期現象はホタルの明滅や神経細胞,心筋細胞など自然界で広く見られ,また,電力網などの工学系でも見られる.
同期現象は注目する周期的な現象に対して定義される位相が一致する現象として数理的には理解される.
そのような同期現象を記述する数理モデルは様々なものが提案されているが,特に蔵元位相振動子モデル(以下,蔵元モデル)は研究が盛んに行われている.
蔵元モデルは,それぞれ固有の振動数に従う自律的な振動子がそれぞれの位相差の$\sin$関数で結合しているとするモデルである.
蔵元モデルは様々な興味深い性質を持つが,臨界結合強度というそれより結合強度が大きいと同期が実現するような結合強度で起きる相転移現象は特に注目されている.
また,蔵元モデルの進歩と同時に,ネットワーク科学が発展し,現実のネットワークの特徴に対して示唆を与えられ,力学系に新たな観点が導入された.
接続次数などのネットワークの構造がパーコレーションや同期,拡散などの動的なプロセスにどう影響しているのかという疑問は多くの研究者を惹きつけるところとなった.
従来のモデルに加えて,次数分布や部分グラフの分布といったトポロジカルな性質を考慮したモデルが提案され,動的なプロセスとの関係が調べられている.
蔵元モデルうううう\cite{RODRIGUES20161}
\section{研究動機}
\label{chap:intro-motiv}
位相振動子として解釈可能な系において,パーコレーションのような構造的な性質が系自体の機能に関係することが分かっている.

\section{本論文の構成}
\label{chap:intro-config}
\end{document}