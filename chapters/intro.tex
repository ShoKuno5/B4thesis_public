\documentclass[../main]{subfiles}
\begin{document}
\chapter{はじめに}
\section{背景}
\label{chap:intro-back}
同期現象はホタルの明滅や神経細胞,心筋細胞など自然界で広く見られ\cite{Strogatz1993},また,電力網などの工学系でも見られる\cite{Dorfler2012}.
同期現象は注目する周期的な現象に対して定義される振動数が部分的に一致する現象として数理的には理解される.
そのような同期現象を記述する数理モデルは様々なもの\cite{Winfree1967,kuramoto1975}が提案されているが,特に蔵本位相振動子モデル(以下,蔵本モデル)は研究が盛んに行われている\cite{RevModPhys.77.137}.
蔵本モデルは,それぞれ固有の振動数に従う自律的な振動子がそれぞれの位相差の$\sin$関数で結合しているとするモデルである.
蔵本モデルは様々な興味深い性質を持つが,臨界結合強度というそれより結合強度が大きいと同期が実現するような結合強度で起きる相転移現象は特に注目されている\cite{RevModPhys.77.137,RODRIGUES20161}.
また,ネットワーク科学の発展により,現実のネットワークの特徴に対して示唆を与えられ,力学系に新たな観点が導入された\cite{RODRIGUES20161}.
接続次数などのネットワークの構造が同期,拡散などの動的なプロセスにどう影響しているのかという疑問は多くの研究者を惹きつけるところとなった.
従来のモデルに加えて,次数分布や部分グラフの分布といったトポロジカルな性質を考慮したモデルが提案され,動的なプロセスとの関係が調べられている\cite{Moreno_2004,PhysRevLett.106.128701,Liu2013,PhysRevLett.124.218301}.

興奮系からなるネットワークにおいて,percolation のような構造的な性質がネットワーク自体の機能に関係することが分かっている\cite{PhysRevLett.110.158101,Pasquale2008SelforganizationAN}.
例えば,不整脈を起こす心臓組織は損傷などにより興奮性の細胞が不均一に結合し構成されている.
この不均一性により自律的な電気回路が部分的に形成され,電気信号が不均一領域と相互作用することにより不規則に異常な興奮が繰り返し発生する(リエントリー).
そして,リエントリーの発生には,不均一領域の percolation を起こす閾値が大きな影響を及ぼしているとされている\cite{PhysRevLett.110.158101}.

しかしながら,percolation と同期現象とを同時に取り扱う研究は十分になされていない.
同期現象を理解する場合それぞれの振動子の振動数が主要な役割を果たすため,興奮現象などの同期現象を理解する上ではネットワークの構造の一方だけを考慮することは十分とは言えない.
そのため,ネットワークにおける振動数分布がその構造とどのように関係するか調べることは,ネットワークの性質をさらに深く理解する助けになることが期待される.

以上を背景として,ネットワークの同期現象を調べる上でよく用いられる振動子ネットワークについて,ネットワーク構造と同期状態やクラスタ状態の関係を調べる.
注目するネットワーク構造として percolation で着目するネットワークに存在する枝の本数を採用し,
枝の生成・消滅に伴う同期状態の変化という細かい性質を調べる.
続いて,枝の本数と振動数によるクラスタ状態の平均的な関係という一般的な性質について調べる.
\section{本論文の構成}
\label{chap:intro-config}
本論文では,全5章からなる.続く第\ref{chap:prev}章では本論文を読むにあたって必要となる前提知識について述べる.
第\ref{chap:method-3body}章では振動子ネットワークにおいて枝を生成・消滅させたときの同期状態の変化について調べる.
第\ref{chap:percolation}章では振動子ネットワークにおいて枝の数を変化させたときの平均的な振動数によるクラスタ状態の変化について調べる.
そして,第\ref{chap:summary}章では本論文をまとめる.
\end{document}