\documentclass[../main]{subfiles}
\begin{document}
\chapter{付録}
\label{chap:appendix}
\cite{XiaHuang:130506}の解析内容について述べる.\\
鎖状に無向グラフとして$N$個連なった位相振動子ネットワークのふるまいを以下で記述するとする.
\begin{align*}
    \dot{\theta}_1&=\omega_1\\
    \dot{\theta}_i&=\omega_i+\lambda\sin(\theta_{i-1}-\theta_i),\ i=2,3,\ldots,N
\end{align*}
ここで,$\theta_i,\omega_i$をそれぞれ端から$i$番目の振動子の位相と固有振動数とする.$\lambda>0$は振動子間の結合強度とする.
このモデルでは端の振動子$1$が固有振動数$\omega_1$をもつ駆動源となり,他の振動子は駆動源に近い方の隣接した振動子と相互作用する.\\

$\omega_1=2,\omega_2=1$とし,他の$\omega_i$をランダムに選択するとして固有振動数を配置し,結合強度を高めていく数値実験を行ったところ,4つのクラスタリングパターンが観測された.
\begin{itemize}
    \item 隣接した駆動振動子同士が同期し局所クラスターが発生する.
    \item 駆動源と同期する振動子が出現する.
    \item 結合強度が小さい間は隣接した駆動振動子と同期し局所クラスターを構成した後,結合強度がある程度大きくなると局所クラスターが消滅する.
    \item 隣接した駆動振動子が駆動源と同期し,駆動される振動子自体も同期する.
\end{itemize}
つまり,各振動子とクラスターを構成するのは,隣接した駆動振動子と,駆動源の2つの振動子のみである.
このパターンは,他の鎖状ネットワークでも普遍的であるため,これらのクラスタリングパターンの分岐を考える.

解析のため,特に$N=3,\ \omega_1=2,\ \omega_2=1$として
\end{document}