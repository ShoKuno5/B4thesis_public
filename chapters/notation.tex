\documentclass[../main]{subfiles}
\begin{document}
\chapter*{本論文で使用する記号}
\begin{itemize}
    \item
    $\mathbb{R}$:実数全体からなる集合.
    \item
    $\mathbb{Z}$:整数全体からなる集合.
    \item
    $\mathbb{S}^n$:$n$次元球面.以下で定義される.
    \begin{equation*}
        \mathbb{S}^n:=\{x\in\mathbb{R}^{n+1}|x_1^2+x_2^2+\cdots+x_{n+1}^2=1\}
    \end{equation*}    
    \item
    $\mathbb{T}^n$:$n$次元トーラス.以下で定義される.
    \begin{equation*}
        \mathbb{T}^n:=\mathbb{S}^1\times\mathbb{S}^1\times \cdots\times \mathbb{S}^1
    \end{equation*}
    \item 
    $A$:グラフの隣接行列.
    \item
    $f(x)=O(g(x))$:$x$が十分大きいとき$f(x)$が$g(x)$以下に抑えられること.つまり,
    \begin{equation*}
        \lim_{x\to\infty}f(x)/g(x)=0
    \end{equation*}
    \item
    $f(x)=\omega(g(x))$:$x$が十分大きいとき$g(x)$が$f(x)$以下に抑えられること.つまり,
    \begin{equation*}
        \lim_{x\to\infty}g(x)/f(x)=0
    \end{equation*}
    \item
    $f(x)\sim g(x)$:$x$が十分大きいとき$f(x)$が$g(x)$にほとんど等しいこと.つまり,
    \begin{equation*}
        \lim_{x\to\infty}f(x)/g(x)=1
    \end{equation*}    
    \item
    $f(x)\gtrsim g(x)$,$g(x)\lesssim f(x)$:$f(x)\sim g(x)$または$f(x)>g(x)$.
    \item
    $f(x)\gtrapprox g(x)$,$g(x)\lessapprox f(x)$:任意の$x$で,$f(x)>g(x)$だが$f(x)$と$g(x)$は十分近いこと.
    \item
    $\dot{x}(s)$:$\mathrm{d}x/\mathrm{d}t|_{t=s}$.
    \item
    $\imag$:虚数単位.
    \item
    $G(V,E)$:頂点集合$V$,枝集合$E$とするグラフ$G$.
    \item
    $x\gg y$:$x$が$y$と比べて十分大きいこと.本論文では$y/x$を微小量として近似を行うことに利用する.
    \item
    $\operatorname{sgn}(x)$:$x\in\mathbb{R}$に対し,$x$の符号を返す関数.
\end{itemize}
\end{document}