\section{背景}
\begin{frame}{LD Cycleによる生体リズムの研究\red{technical}}
        \begin{block}{概日リズムの研究:LD Cycle}
          \begin{itemize}
            \item 時差ぼけやシフトワークは、生活習慣病のリスクを高める.
            \item LD Cycleは生体リズムのモデルに用いられ\red{おかしい},光と暗闇の情報が外力として系に加わる.\begin{itemize}
              \item 時差ぼけ/シフトワークの研究では,この外力の位相をシフトさせることを考える.
            \end{itemize}
            \item 外力や各パラメータの設定により周期的・カオス的な振る舞いを持つ.
            \item 一般にカオスの予測は難しい(初期値鋭敏性).
          \end{itemize}
        \end{block}
        \vspace{-.5em}
        \begin{figure}
          \includegraphics[width=0.4\textwidth]{Fig/Jetlag.png}
          \caption{\scriptsize{Day 1に8時間のJet Lagを受けた時のマウスの Per2 の推移}\\\tiny{Image: Fig.2 from \cite{Yamaguchi et al.}.}}
      \end{figure}    
      
  \end{frame}

  \section{手法}

\begin{frame}{Reservoir Computer の特徴}
    % スライドを2つの列に分割
    \begin{columns}[T] % [T] は列を上部で揃えるオプション
  
      \begin{column}{.5\textwidth}
        \begin{block}{Reservoir Computer}
          input, reservoir, output層から成るRecurrent Neural Network(RNN) の一種.特徴は,
          \begin{itemize}
            \item 入力を高次元の reservoir 層に写す.\begin{itemize}
              \item reservoir 内のニューロンはランダムに接続されている.
              \item 高次元のダイナミクスを持ち,複雑な時系列パターンを捉えやすい.
            \end{itemize}
            \item 出力層のみを学習\begin{itemize}
              \item 計算コストが低く,過学習のリスクも減少する.
            \end{itemize}
            \item 時間を通じて過去の入力の影響を保持する.
          \end{itemize}
        \end{block}
      \end{column}

      \begin{column}{.5\textwidth}
        \red{fbがないのに}

        \begin{figure}
            \includegraphics[width=\textwidth]{Fig/esn.svg.png}
        \end{figure}  
        \begin{figure}
            \includegraphics[width=\textwidth]{Fig/esn_nodes.svg.png}
            \caption{\scriptsize{Reservoirpy での reservoir computer の構造}\\ \tiny{Image: ReservoirPy, MIT License.}}
        \end{figure}  
      \end{column}
    \end{columns}
  \end{frame}


