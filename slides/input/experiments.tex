\begin{frame}{周期外力のあるRösslerモデル (1/2)}
    \vspace{-.35cm}
    % スライドを2つの列に分割
    \begin{columns}[T] % [T] は列を上部で揃えるオプション
  
      \begin{column}{.5\textwidth}
        \begin{block}{Rössler方程式}
            次の方程式を解くことで時系列データを生成する.
            \begin{align}
                \frac{dx}{dt} &= -y - z + P(t)\\
                \frac{dy}{dt} &= x + ay \\
                \frac{dz}{dt} &= b + z(x - c)
            \end{align}
            ここで,$P(t) := A \sin(t + \theta_p(t))$ とする.
           
            ただし,$p \in \left\{ n \in \Z \mid -12 \leq n \leq 12 \right\}$ とし,
            $\theta_p$ は $4$ 日に $1$ 度外力の位相を $p\cdot 2\pi/24$ だけ早めるような関数.
        \end{block}
      \end{column}
      \begin{column}{.5\textwidth}
        \vspace{-.6cm}
        \begin{figure}
            \includegraphics[width=0.9\textwidth]{Fig/NEWrossler_attractor.png}
            \caption{\scriptsize{Rösslerモデル}}
        \end{figure}
        
      \end{column}
    \end{columns}
  \end{frame}

\begin{frame}{周期外力のあるRösslerモデル (2/2)}
    \begin{columns}[T] % [T] は列を上部で揃えるオプション
        \begin{column}{.5\textwidth}
            \begin{itemize}
                \item Hyperparameter の最適化は位相シフトが$8$時間である系に対して行う.
                \item その後,同じ Reservoir で異なる位相シフトを持つ系に対しても予測を行う.\begin{itemize}
                    \item 未知の外力に対するReservoir の予測性能を測る.
                \end{itemize}
            \end{itemize}
            
            \vspace{-.5em}
            \begin{figure}
                %\centering % 画像を中央揃えにする(オプション)
                \includegraphics[width=0.8\textwidth]{Fig/phase_shift_plot.png}
                \caption{\scriptsize{$4$ 日に一度 $8$ 時間早めたときの外力の位相}}
            \end{figure}
        \end{column}
        \begin{column}{.5\textwidth}
            \vspace{-1cm}
            \begin{figure}
                %\centering % 画像を中央揃えにする(オプション)
                \includegraphics[width=0.8\textwidth]{Fig/NEWrossler_waves.png}
                \caption{\scriptsize{位相シフトのある周期外力付きのRössler システム:\\
                上から $x, y, z, P(t)$. }}
            \end{figure}
        \end{column}
      \end{columns}
    
\end{frame}


\section{結果}
\begin{frame}
    
\end{frame}

\section{まとめと展望}
\begin{frame}
    
\end{frame}
