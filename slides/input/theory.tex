\section{理論的な展開}

\begin{frame}{理論的な展開}
    Resrvoir Computerの理論的な研究としては Bollt (2021) の研究などが挙げられる.
    \begin{block}{Bollt (2021)}
        Reservoir Computer の activation function (式 \eqref{Lai_r1}, \eqref{Lai_r2} における $\tanh$ 関数)を局所的に線型 activation function $q:\R \ni s \mapsto s \in \R$ とみなすことで,Reservoir 内のダイナミクスに関して次のを得る. \vspace{-.5cm}
        \begin{align}
            \mathbf{v}_{l+1} = \mathbf{W}^{\text {out }} \sum_{j=1}^{\ell} \mathbf{A}^{j-1} \mathbf{W}^{\text {in }} \mathbf{u}_{\ell-j+1} \label{Bollt_eq}
        \end{align}
        式\eqref{Bollt_eq} が VAR(NVAR) の係数行列を表すものだとみなせる.
    \end{block}

\end{frame}
