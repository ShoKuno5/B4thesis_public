\documentclass[uplatex]{suribt}
%\documentclass[oneside]{suribt}% 本文が * ページ以下のときに (掲示に注意)

\usepackage{hyperref}

\title{Reservoir Computer による\\カオス時系列予測と\\生体リズム研究への応用}
%\titlewidth{}% タイトル幅 (指定するときは単位つきで)
\author{久野 証}
\eauthor{Sho Kuno}% Copyright 表示で使われる
\studentid{03-210599}
\supervisor{郡 宏 教授}% 1 つ引数をとる (役職まで含めて書く)
%\supervisor{指導教員名 役職 \and 指導教員名 役職}% 複数教員の場合,\and でつなげる
\handin{2024}{2}% 提出月. 2 つ (年, 月) 引数をとる
%\keywords{キーワード1, キーワード2} % 概要の下に表示される


\begin{document}
\maketitle%%%%%%%%%%%%%%%%%%% タイトル %%%%

\frontmatter% ここから前文
\begin{abstract}%%%%%%%%%%%%% 概要 %%%%%%%%
 ここに概要を書く.
\end{abstract}

\tableofcontents%%%%%%%%%%%%% 目次 %%%%%%%%

\mainmatter% ここから本文 %%% 本文 %%%%%%%%

\subsection{INTRODUCTION}
\begin{itemize}
    \item Digital twins: through mathematical modeling or data.
    \item Reservoir Computers: self-dynamical evolution with memory / no back-propagation.
    \item $\bold{novelty}$: "Introducing a
    control mechanism into the RC structure with an exoge-
    nous control signal acting directly onto the RC network
    distinguishes our work from existing ones in the litera-
    ture of RC as applied to nonlinear dynamical systems."
    \item $\bold{Purpose}$: "Of particular interest is whether the collapse of the tar-
    get chaotic system can be anticipated from the digital
    twin. The purpose of this paper is to demonstrate that
    the digital twin so created can accurately produce the
    bifurcation diagram of the target system and faithfully
    mimic its dynamical evolution from a statistical point of
    view. The digital twin can then be used to monitor the
    present and future “health” of the system. More impor-
    tantly, with proper training from observational data the
    twin can reliably anticipate system collapses, providing
    early warnings of potentially catastrophic failures of the
    system"
    \item $\bold{Targets}$: \begin{enumerate}
        \item extrapolation of the dynamical evolution of the target
        system into certain “uncharted territories” in the param-
        eter space
        \item long-term continual forecasting of nonlin-
        ear dynamical systems subject to non-stationary external
        driving with sparse state updates
        \item inference of hidden variables in the system and accurate prediction of their dynamical evolution into the future
        \item adaptation to external driving of different waveform
        \item extrapo-lation of the global bifurcation behaviors of network sys-tems to some different sizes. 
    \end{enumerate}
\end{itemize}

\clearpage

\chapter{前提知識}

\section{生体リズム研究}
\section{力学系}
\section{Van Der Polモデル}
\section{Reservoir Computer}
\section{先行研究}

\clearpage

\subsection{METHODS}
\begin{itemize}
    \item Digital twins: a recurrent RC neural network with a control
    mechanism, which requires two types of input signals:
    the observational time series for training and the con-trol signal $f(t)$ that remains in both the training and
    self-evolving phase. \begin{itemize}
        \item During the train-ing, the hidden recurrent layer is driven by both the in-
        put signal $u(t)$ and the control signal $f(t)$. 
    \end{itemize}
    \item Reservoir updating equations:
    \begin{align}
    \text{training phase: }& \mathbf{r}(t+\Delta t)=(1-\alpha) \mathbf{r}(t) + \alpha \tanh \left[\mathcal{W}_r \mathbf{r}(t)+\mathcal{W}_{\text {in }} \mathbf{u}(t)+\mathcal{W}_c f(t)\right] \\
    \text{self-evolving phase: } & \mathbf{r}(t+\Delta t)=(1-\alpha) \mathbf{r}(t) + \alpha \tanh \left[\mathcal{W}_r \mathbf{r}(t)+\mathcal{W}_{\text {in }} \mathcal{W}_{\text {out }} \mathbf{r}^{\prime}(t)+\mathcal{W}_c f(t)\right]
    \end{align}

    \item $\bold{“sense, learn, and mingle”}$: During
    the training, several trials of data are typically used un-der different driving signals so that the digital twin can
    the responses of the target sys-tem to gain the ability to extrapolate a response to a new driving signal that has never been encountered before. We input these trials of training data, i.e., a few pairs of $\mathbf{u}(t)$ and the associated $f(t)$, through the matrices $\mathcal{W}_{\text {in }}$ and $\mathcal{W}_c$ sequentially. 

    \item $\bold{validation/testing}$: the validation of the RC net-works are done with the same driving signals $f(t)$ as in
    the training data. We test driving signals $f(t)$ that are
    different from those generating the training data (e.g.,
    with different amplitude, frequency, or waveform).  

    \item $\bold{warming-up}$ During the warming-up process to initialize the RC networks prior to making the predictions, we feed randomly chosen short segments of the training time series to feed into the RC network.
    That is, no data from the target system under the testing
    driving signals $f(t)$ are required for making the predic-
    tions.
\end{itemize}


\clearpage


\chapter{結果}

\section{Van Der Pol モデル}

\section{Rössler モデル}

\clearpage

\chapter{議論}
\clearpage

\cite{RODRIGUES20161}

\backmatter% ここから後付
\chapter{謝辞}%%%%%%%%%%%%%%% 謝辞 %%%%%%%

%\begin{thebibliography}{}%%%% 参考文献 %%%
% \bibitem{}
%\end{thebibliography}
\bibliographystyle{unsrt}%           BibTeX を使う場合
\bibliography{reference.bib}% BibTeX を使う場合

\appendix% ここから付録 %%%%% 付録 %%%%%%%
\chapter{}
\end{document}