\documentclass[uplatex]{suribt}
%\documentclass[oneside]{suribt}% 本文が * ページ以下のときに (掲示に注意)

\usepackage{hyperref}

\title{Reservoir Computer によるカオス時系列予測と生体リズム研究への応用}
%\titlewidth{}% タイトル幅 (指定するときは単位つきで)
\author{久野 証}
\eauthor{Sho Kuno}% Copyright 表示で使われる
\studentid{03-210599}
\supervisor{郡 宏 教授}% 1 つ引数をとる (役職まで含めて書く)
%\supervisor{指導教員名 役職 \and 指導教員名 役職}% 複数教員の場合,\and でつなげる
\handin{2024}{2}% 提出月. 2 つ (年, 月) 引数をとる
%\keywords{キーワード1, キーワード2} % 概要の下に表示される


\begin{document}
\maketitle%%%%%%%%%%%%%%%%%%% タイトル %%%%

\frontmatter% ここから前文
\begin{abstract}%%%%%%%%%%%%% 概要 %%%%%%%%
 ここに概要を書く.
\end{abstract}

\tableofcontents%%%%%%%%%%%%% 目次 %%%%%%%%

\mainmatter% ここから本文 %%% 本文 %%%%%%%%
\chapter{}

\cite{RODRIGUES20161}

\backmatter% ここから後付
\chapter{謝辞}%%%%%%%%%%%%%%% 謝辞 %%%%%%%

%\begin{thebibliography}{}%%%% 参考文献 %%%
% \bibitem{}
%\end{thebibliography}
\bibliographystyle{unsrt}%           BibTeX を使う場合
\bibliography{reference.bib}% BibTeX を使う場合

\appendix% ここから付録 %%%%% 付録 %%%%%%%
\chapter{}
\end{document}