% common_preamble.tex
% ==================== 共通プリアンブル ====================


% パッケージの読み込みと設定
\usepackage{lmodern}

% ページ数表記設定
\usepackage{fancyhdr}
\pagestyle{fancy}
\fancyhf{} % clear all header and footer fields
\fancyfoot[C]{\thepage} % ページ番号を中央下に配置
\renewcommand{\headrulewidth}{0pt} % ヘッダの線を表示しない
\renewcommand{\footrulewidth}{0pt} % フッタの線を表示しない

% 余白設定
\usepackage[top=30truemm,bottom=30truemm,left=20truemm,right=20truemm]{geometry}

% リンク設定
\usepackage{hyperref}
\hypersetup{
    colorlinks=true,
    citecolor=blue,
    linkcolor=blue,
}

% tcolorbox 設定
\usepackage[most]{tcolorbox}
\tcbuselibrary{breakable, skins, theorems}

% その他のパッケージと設定
\usepackage{xcolor}
\usepackage{ulem}
\usepackage{amssymb,amsfonts, amsthm}
\usepackage{mathtools}
\mathtoolsset{showonlyrefs=true}
\usepackage{booktabs,tabularx}
\usepackage{url}
\usepackage{graphicx}
\usepackage{autobreak}
\usepackage{braket,physics}
\usepackage{bm}
\usepackage{latexsym}
\usepackage{comment}
\usepackage{pdfpages}
\usepackage{here}
\usepackage{multicol}
\setlength{\columnsep}{5mm}
\columnseprule=0.2mm
\usepackage{ascmac}
\usepackage{array}
\usepackage{enumitem}
\usepackage{lastpage}
\usepackage{cancel}

% 数学の基本的な定義やショートカット
\newcommand{\R}{\mathbb{R}}
\newcommand{\Z}{\mathbb{Z}}
\newcommand{\N}{\mathbb{N}}
\newcommand{\C}{\mathbb{C}}
\newcommand{\E}{\mathbb{E}}
\newcommand{\F}{\mathcal{F}}
\newcommand{\B}{\mathcal{B}}
\newcommand{\oo}{\omega}
\newcommand{\OO}{\Omega}


% 前書きのスタイル
\newcommand{\prefacename}{Preface}
\newenvironment{preface}{
    %\vspace*{\stretch{2}}
    {\noindent \bfseries \Huge \prefacename}\\[-3.5em] 
    \begin{center}
        %\phantomsection \addcontentsline{toc}{chapter}{\prefacename} % enable this if you want to put the preface in the table of contents
        \thispagestyle{plain}
    \end{center}%
}
{}

% 目次のスタイル
\renewcommand{\contentsname}{} % ここでデフォルトのタイトルを空に設定します
\newenvironment{customcontents}{
    % 目次ページの開始
    {\noindent \bfseries \Huge Contents}\\[-3.5em]
    \begin{center}
        % \phantomsection \addcontentsline{toc}{chapter}{Contents} % この行を有効にすると目次自体を目次に含めることができます
        \thispagestyle{plain}
    \end{center}%
    % 目次の表示
    \vspace*{-6.5em}
    \hypersetup{linkcolor=blue} % ここでリンクの色を設定
    \tableofcontents
    \hypersetup{linkcolor=black} % 目次以外のリンクの色を設定
}
{}




% その他のスタイルや設定
\allowdisplaybreaks[4]
\renewcommand{\baselinestretch}{1.5}

% 可換図を書くためのパッケージ
\usepackage{tikz}
\usetikzlibrary{matrix,arrows,decorations.pathmorphing}

% 段落のスタイルの変更
\makeatletter
\renewcommand\paragraph{\@startsection{paragraph}{4}{\z@}%
  {-3.25ex \@plus1ex \@minus.2ex}%
  {1.5ex \@plus.2ex}%
  {\normalfont\normalsize\bfseries}}
\makeatother

% グラフィックスの設定 (お勧め)
\graphicspath{{images/}} % 画像のディレクトリを指定

% 一般的なスタイルと微調整
\renewcommand{\sectionmark}[1]{\markboth{#1}{}} 
\renewcommand{\subsectionmark}[1]{\markright{#1}}

% セクションの定義を変更
\let\oldsection\section
\renewcommand{\section}{\thispagestyle{plain}\oldsection}

% ソースコードの表示設定
\usepackage{listings}
\lstset{
  language=[LaTeX]TeX,
  breaklines=true,
  basicstyle=\ttfamily,
  keywordstyle=\color{blue},
  stringstyle=\color{orange},
}
\makeatletter

% フットノートルールの変更
\newcommand{\customizefootnoterule}{%
  \def\footnoterule{\kern-3\p@
    \hrule \@width \columnwidth % ここを変更
    \kern 2.6\p@}%
}

\customizefootnoterule % ここでマクロを呼び出し
% 番号なしフットノートのマクロ
\newcommand{\blankfootnote}[1]{%
  \begingroup
  \let\@makefnmark\empty % \@makefnmark を空にする
  \footnotetext{#1}%
  \endgroup
}

\makeatother


\usepackage{xparse}

\newcommand{\Cinfty}{\text{C}^\infty}


%lecture1.tex
% ==================== 各レクチャー固有の設定 ====================

% レクチャー番号の設定
\newcounter{lecture}
\setcounter{lecture}{1} % Xは各レクチャーの番号
% 各レクチャーの開始前にカウンターを増加させる
\newcommand{\nextlecture}{
    \stepcounter{lecture}  % Increment the lecture counter here
}


\newcommand{\lecturetitle}[1]{
    {
        \phantomsection  % Add this to fix the hyperref issue
        \centering  % centering without additional vertical space
        LECTURE \arabic{lecture} \\[-1em]  % Adjust vertical space
        \LARGE \textbf{#1} \\
    }
    \addcontentsline{toc}{section}{LECTURE \arabic{lecture}: #1}  % Add to ToC
    \vspace{3em}
}

\newcounter{item}
% item カウンターの表示形式を lecture.item に設定
\renewcommand{\theitem}{\arabic{lecture}.\thesubsection.\arabic{item}}

\makeatletter
\@addtoreset{item}{subsection}
\makeatother


% 各環境に対する新しいラベリングシステム
\NewDocumentEnvironment{definition}{o}{
  \refstepcounter{item}%
  \par\noindent{\large D}EFINITION \theitem\quad
  \IfValueT{#1}{\label{#1}}
}{\par}

% 定理環境
\NewDocumentEnvironment{theorem}{o}{
  \refstepcounter{item}%
  \par\noindent{\large T}HEOREM \theitem\quad
  \IfValueT{#1}{\label{#1}}
}{\par}

% 補題環境
\NewDocumentEnvironment{lemma}{o}{
  \refstepcounter{item}%
  \par\noindent{\large L}EMMA \theitem\quad
  \IfValueT{#1}{\label{#1}}
}{\par}

% 注意環境
\NewDocumentEnvironment{remark}{o}{
  \refstepcounter{item}%
  \par\noindent{\large R}EMARK \theitem\quad
  \IfValueT{#1}{\label{#1}}
}{\par}

% 命題環境
\NewDocumentEnvironment{proposition}{o}{
  \refstepcounter{item}%
  \par\noindent{\large P}ROPOSITION \theitem\quad
  \IfValueT{#1}{\label{#1}}
}{\par}

% 系環境
\NewDocumentEnvironment{corollary}{o}{
  \refstepcounter{item}%
  \par\noindent{\large C}OROLLARY \theitem\quad
  \IfValueT{#1}{\label{#1}}
}{\par}

% 練習問題環境
\NewDocumentEnvironment{exercise}{o}{
  \refstepcounter{item}%
  \par\noindent{\large E}XERCISE \theitem\quad
  \IfValueT{#1}{\label{#1}}
}{\par}

% 式環境
\NewDocumentEnvironment{myequation}{o}{
  \refstepcounter{item}%
  \par\noindent{\large E}QUATION \theitem\quad
  \IfValueT{#1}{\label{#1}}
}{\par}

% 例環境
\NewDocumentEnvironment{example}{o}{
  \refstepcounter{item}%
  \par\noindent{\large E}XAMPLE \theitem\quad
  \IfValueT{#1}{\label{#1}}
}{\par}
