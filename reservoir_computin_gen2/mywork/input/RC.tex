\section{Papers}
\subsection{Emergence of a resonance in machine learning}
Zheng-Meng Zhai , 1 Ling-Wei Kong , 1 and Ying-Cheng Lai 1,2,*
1School of Electrical, Computer and Energy Engineering, Arizona State University, Tempe, Arizona 85287, USA
2Department of Physics, Arizona State University, Tempe, Arizona 85287, USA.

\noindent (Received 9 June 2022; revised 1 March 2023; accepted 26 July 2023; published 24 August 2023)
\\

\noindent 勉強のためのキーワード:Resonance in nonlinear dynamical systems, 

\subsubsection{要旨}
\begin{enumerate}
  \item 問い:What is the physical or dynamical mechanism underlying the benefit of noise and
  how do we find the optimal level of noise?
  \item この論文の意義:In this paper, we uncover a resonance phenomenon in
  which a certain amount of noise can significantly enhance
  the short-term and long-term prediction accuracy and robust-
  ness for chaotic systems, where the optimal noise level can
  be found through a generalized scheme of hyperparameter
  \item 難点:A challenging issue is to identify the underlying dynamical mechanism responsible for the emergence of a resonance
  in machine learning optimization.
\end{enumerate}

\subsubsection{状況設定}

\subsubsection{面白いと思ったところ}

\subsubsection{論文を受けての今後の研究方向}
\subsubsection{疑問点}
\begin{enumerate}
  \item Fig.4とFig.5について,MG system における$\tau$の値が30から17に変えると,$\sigma$にどのような影響があるか.なぜその影響が生まれるか.
  \item なぜ,Fig.2の(逆)ピークを与える$\sigma$帯とFig.6(c)の(逆)ピークを与える$\sigma$帯が重なるのか.
  \item IIIで,Machine learningにおけるresonanceが生まれるPhysical reasonを挙げているが,これは対象を正しく説明できているか.extraordinarily complicatedなhidden layerの中身を解析することなく,physical reasonを与えることが,なにを説明しているのか/なにを説明していないのか.
\end{enumerate}

\subsubsection{関連する文献}
 
\clearpage
\subsubsection{用語まとめ}

\subsubsection{Abstract}
\begin{enumerate}
  \item stochastic/coherence resonance: 
  \item nonlinear dynamical system:
  \item regularizer/regularization:
  \item reservoir computing:
  \item state variables/attractor:
  \item hyperparameters:
\end{enumerate}
\subsubsection{I. Introduction}
\begin{enumerate}
  \item model-free/data-driven:
  \item oscillatoin/Lyapnov times:
  \item trajectory:
  \item basin boundary: 
  \item robustness:
  \item Baysian optimization:
\end{enumerate}

\subsubsection{II. Result}
\begin{enumerate}
  \item SURROGATEOPT function (MATLAB):
  \item surrogate approximation function:
  \item objective function:
  \item global minimum:
  \item sampling/updating:
  \item radial basis function: 
  \item Mackey-Glass (MG) system:
  \item spatiotemporal chaotic Kuramoto-Sivashinsky (KS) system: 
\end{enumerate}

\subsubsection{A. Emergence of a resonance from short-term prediction}
\begin{enumerate}
  \item transient behavior:
  \item z-score normalization: 
  \item periodic boundary condition:
  \item Prediction horizon/stability: 
\end{enumerate}

\subsubsection{B. Emergence of a resonance from long-term prediction}
\begin{enumerate}
  \item collapse:
  \item wider/narrower resonance:
\end{enumerate}

\subsubsection{III. HEURISTIC REASON FOR THE OCCURRENCE
OF A RESONANCE}
\begin{enumerate}
  \item time-scale match:
  \item the mean first-passage time:
  \item nonlinear activation: 
  \item linear reservoir computing:
  \item noise-enhanced temporal regularity:
  \item vector autoregressive process (VAR): 
\end{enumerate}

\subsubsection{IV. DISCUSSION}
\begin{enumerate}
  \item magnitude: 
\end{enumerate}

\subsubsection{Appendix A}
\begin{enumerate}
  \item recurrent neural network(RNN):
  \item input/hidden/output layer: 
  \item linear regression: 
  \item adjacency matrix:
  \item state vector: 
  \item dynamical state/evolution: 
  \item neuron:
  \item leakage parameter $\alpha$:
  \item link probability p: 
  \item spectral radius:
\end{enumerate}

\clearpage